% Options for packages loaded elsewhere
\PassOptionsToPackage{unicode}{hyperref}
\PassOptionsToPackage{hyphens}{url}
%
\documentclass[
]{article}
\usepackage{amsmath,amssymb}
\usepackage{lmodern}
\usepackage{iftex}
\ifPDFTeX
  \usepackage[T1]{fontenc}
  \usepackage[utf8]{inputenc}
  \usepackage{textcomp} % provide euro and other symbols
\else % if luatex or xetex
  \usepackage{unicode-math}
  \defaultfontfeatures{Scale=MatchLowercase}
  \defaultfontfeatures[\rmfamily]{Ligatures=TeX,Scale=1}
\fi
% Use upquote if available, for straight quotes in verbatim environments
\IfFileExists{upquote.sty}{\usepackage{upquote}}{}
\IfFileExists{microtype.sty}{% use microtype if available
  \usepackage[]{microtype}
  \UseMicrotypeSet[protrusion]{basicmath} % disable protrusion for tt fonts
}{}
\makeatletter
\@ifundefined{KOMAClassName}{% if non-KOMA class
  \IfFileExists{parskip.sty}{%
    \usepackage{parskip}
  }{% else
    \setlength{\parindent}{0pt}
    \setlength{\parskip}{6pt plus 2pt minus 1pt}}
}{% if KOMA class
  \KOMAoptions{parskip=half}}
\makeatother
\usepackage{xcolor}
\usepackage[margin=1in]{geometry}
\usepackage{color}
\usepackage{fancyvrb}
\newcommand{\VerbBar}{|}
\newcommand{\VERB}{\Verb[commandchars=\\\{\}]}
\DefineVerbatimEnvironment{Highlighting}{Verbatim}{commandchars=\\\{\}}
% Add ',fontsize=\small' for more characters per line
\usepackage{framed}
\definecolor{shadecolor}{RGB}{248,248,248}
\newenvironment{Shaded}{\begin{snugshade}}{\end{snugshade}}
\newcommand{\AlertTok}[1]{\textcolor[rgb]{0.94,0.16,0.16}{#1}}
\newcommand{\AnnotationTok}[1]{\textcolor[rgb]{0.56,0.35,0.01}{\textbf{\textit{#1}}}}
\newcommand{\AttributeTok}[1]{\textcolor[rgb]{0.77,0.63,0.00}{#1}}
\newcommand{\BaseNTok}[1]{\textcolor[rgb]{0.00,0.00,0.81}{#1}}
\newcommand{\BuiltInTok}[1]{#1}
\newcommand{\CharTok}[1]{\textcolor[rgb]{0.31,0.60,0.02}{#1}}
\newcommand{\CommentTok}[1]{\textcolor[rgb]{0.56,0.35,0.01}{\textit{#1}}}
\newcommand{\CommentVarTok}[1]{\textcolor[rgb]{0.56,0.35,0.01}{\textbf{\textit{#1}}}}
\newcommand{\ConstantTok}[1]{\textcolor[rgb]{0.00,0.00,0.00}{#1}}
\newcommand{\ControlFlowTok}[1]{\textcolor[rgb]{0.13,0.29,0.53}{\textbf{#1}}}
\newcommand{\DataTypeTok}[1]{\textcolor[rgb]{0.13,0.29,0.53}{#1}}
\newcommand{\DecValTok}[1]{\textcolor[rgb]{0.00,0.00,0.81}{#1}}
\newcommand{\DocumentationTok}[1]{\textcolor[rgb]{0.56,0.35,0.01}{\textbf{\textit{#1}}}}
\newcommand{\ErrorTok}[1]{\textcolor[rgb]{0.64,0.00,0.00}{\textbf{#1}}}
\newcommand{\ExtensionTok}[1]{#1}
\newcommand{\FloatTok}[1]{\textcolor[rgb]{0.00,0.00,0.81}{#1}}
\newcommand{\FunctionTok}[1]{\textcolor[rgb]{0.00,0.00,0.00}{#1}}
\newcommand{\ImportTok}[1]{#1}
\newcommand{\InformationTok}[1]{\textcolor[rgb]{0.56,0.35,0.01}{\textbf{\textit{#1}}}}
\newcommand{\KeywordTok}[1]{\textcolor[rgb]{0.13,0.29,0.53}{\textbf{#1}}}
\newcommand{\NormalTok}[1]{#1}
\newcommand{\OperatorTok}[1]{\textcolor[rgb]{0.81,0.36,0.00}{\textbf{#1}}}
\newcommand{\OtherTok}[1]{\textcolor[rgb]{0.56,0.35,0.01}{#1}}
\newcommand{\PreprocessorTok}[1]{\textcolor[rgb]{0.56,0.35,0.01}{\textit{#1}}}
\newcommand{\RegionMarkerTok}[1]{#1}
\newcommand{\SpecialCharTok}[1]{\textcolor[rgb]{0.00,0.00,0.00}{#1}}
\newcommand{\SpecialStringTok}[1]{\textcolor[rgb]{0.31,0.60,0.02}{#1}}
\newcommand{\StringTok}[1]{\textcolor[rgb]{0.31,0.60,0.02}{#1}}
\newcommand{\VariableTok}[1]{\textcolor[rgb]{0.00,0.00,0.00}{#1}}
\newcommand{\VerbatimStringTok}[1]{\textcolor[rgb]{0.31,0.60,0.02}{#1}}
\newcommand{\WarningTok}[1]{\textcolor[rgb]{0.56,0.35,0.01}{\textbf{\textit{#1}}}}
\usepackage{graphicx}
\makeatletter
\def\maxwidth{\ifdim\Gin@nat@width>\linewidth\linewidth\else\Gin@nat@width\fi}
\def\maxheight{\ifdim\Gin@nat@height>\textheight\textheight\else\Gin@nat@height\fi}
\makeatother
% Scale images if necessary, so that they will not overflow the page
% margins by default, and it is still possible to overwrite the defaults
% using explicit options in \includegraphics[width, height, ...]{}
\setkeys{Gin}{width=\maxwidth,height=\maxheight,keepaspectratio}
% Set default figure placement to htbp
\makeatletter
\def\fps@figure{htbp}
\makeatother
\setlength{\emergencystretch}{3em} % prevent overfull lines
\providecommand{\tightlist}{%
  \setlength{\itemsep}{0pt}\setlength{\parskip}{0pt}}
\setcounter{secnumdepth}{-\maxdimen} % remove section numbering
\ifLuaTeX
  \usepackage{selnolig}  % disable illegal ligatures
\fi
\IfFileExists{bookmark.sty}{\usepackage{bookmark}}{\usepackage{hyperref}}
\IfFileExists{xurl.sty}{\usepackage{xurl}}{} % add URL line breaks if available
\urlstyle{same} % disable monospaced font for URLs
\hypersetup{
  pdftitle={covidreport},
  pdfauthor={PANG KAM HING},
  hidelinks,
  pdfcreator={LaTeX via pandoc}}

\title{covidreport}
\author{PANG KAM HING}
\date{2022-08-06}

\begin{document}
\maketitle

\hypertarget{introduction}{%
\subsection{Introduction}\label{introduction}}

COVID-19 has become a global pandemic. Analyzing the patterns can give a
deeper understanding of the virus. the report cites Johns Hopkins github
site, Importing the time series of COVID-19 case(US and global).

\hypertarget{library}{%
\subsection{Library}\label{library}}

\begin{Shaded}
\begin{Highlighting}[]
\FunctionTok{library}\NormalTok{(tidyverse)}
\end{Highlighting}
\end{Shaded}

\begin{verbatim}
## -- Attaching packages --------------------------------------- tidyverse 1.3.2 --
## v ggplot2 3.3.6     v purrr   0.3.4
## v tibble  3.1.7     v dplyr   1.0.9
## v tidyr   1.2.0     v stringr 1.4.0
## v readr   2.1.2     v forcats 0.5.1
## -- Conflicts ------------------------------------------ tidyverse_conflicts() --
## x dplyr::filter() masks stats::filter()
## x dplyr::lag()    masks stats::lag()
\end{verbatim}

\begin{Shaded}
\begin{Highlighting}[]
\FunctionTok{library}\NormalTok{(tinytex)}
\FunctionTok{library}\NormalTok{(lubridate)}
\end{Highlighting}
\end{Shaded}

\begin{verbatim}
## 
## 载入程辑包:'lubridate'
## 
## The following objects are masked from 'package:base':
## 
##     date, intersect, setdiff, union
\end{verbatim}

\begin{Shaded}
\begin{Highlighting}[]
\FunctionTok{library}\NormalTok{(cowplot)}
\end{Highlighting}
\end{Shaded}

\begin{verbatim}
## 
## 载入程辑包:'cowplot'
## 
## The following object is masked from 'package:lubridate':
## 
##     stamp
\end{verbatim}

\hypertarget{importing-data}{%
\subsection{Importing Data}\label{importing-data}}

\begin{Shaded}
\begin{Highlighting}[]
\NormalTok{URL }\OtherTok{=} \StringTok{\textquotesingle{}https://raw.githubusercontent.com/CSSEGISandData/COVID{-}19/master/csse\_covid\_19\_data/csse\_covid\_19\_time\_series/\textquotesingle{}}
\NormalTok{time\_series\_covid19\_confirmed\_US.csv }\OtherTok{=} \FunctionTok{str\_c}\NormalTok{(URL,}\StringTok{\textquotesingle{}time\_series\_covid19\_confirmed\_US.csv\textquotesingle{}}\NormalTok{)}
\NormalTok{time\_series\_covid19\_confirmed\_global.csv }\OtherTok{=} \FunctionTok{str\_c}\NormalTok{(URL,}\StringTok{\textquotesingle{}time\_series\_covid19\_confirmed\_global.csv\textquotesingle{}}\NormalTok{)}
\NormalTok{time\_series\_covid19\_deaths\_US.csv }\OtherTok{=} \FunctionTok{str\_c}\NormalTok{(URL,}\StringTok{\textquotesingle{}time\_series\_covid19\_deaths\_US.csv\textquotesingle{}}\NormalTok{)}
\NormalTok{time\_series\_covid19\_deaths\_global.csv }\OtherTok{=} \FunctionTok{str\_c}\NormalTok{(URL,}\StringTok{\textquotesingle{}time\_series\_covid19\_deaths\_global.csv\textquotesingle{}}\NormalTok{)}
\NormalTok{time\_series\_covid19\_recovered\_global.csv}\OtherTok{=}\FunctionTok{str\_c}\NormalTok{(URL,}\StringTok{\textquotesingle{}time\_series\_covid19\_recovered\_global.csv\textquotesingle{}}\NormalTok{)}
\NormalTok{CUS}\OtherTok{=}\FunctionTok{read.csv}\NormalTok{(time\_series\_covid19\_confirmed\_US.csv)}
\NormalTok{CGL}\OtherTok{=}\FunctionTok{read.csv}\NormalTok{(time\_series\_covid19\_confirmed\_global.csv)}
\NormalTok{DUS}\OtherTok{=}\FunctionTok{read.csv}\NormalTok{(time\_series\_covid19\_deaths\_US.csv)}
\NormalTok{DGL}\OtherTok{=}\FunctionTok{read.csv}\NormalTok{(time\_series\_covid19\_deaths\_global.csv)}
\end{Highlighting}
\end{Shaded}

\hypertarget{data-cleaning}{%
\subsection{Data cleaning}\label{data-cleaning}}

First,To reduce unnecessary variables, in this report we mainly study
the relationship between latitude , case and Case fatality ratio .

\begin{Shaded}
\begin{Highlighting}[]
\NormalTok{CUS}\OtherTok{\textless{}{-}}\NormalTok{CUS}\SpecialCharTok{\%\textgreater{}\%}\FunctionTok{select}\NormalTok{(}\SpecialCharTok{{-}}\NormalTok{UID,}\SpecialCharTok{{-}}\NormalTok{iso2,}\SpecialCharTok{{-}}\NormalTok{iso3,}\SpecialCharTok{{-}}\NormalTok{code3,}\SpecialCharTok{{-}}\NormalTok{FIPS,}\SpecialCharTok{{-}}\NormalTok{Admin2,}\SpecialCharTok{{-}}\NormalTok{Province\_State,}\SpecialCharTok{{-}}\NormalTok{Country\_Region)}\SpecialCharTok{\%\textgreater{}\%}\FunctionTok{pivot\_longer}\NormalTok{(}\AttributeTok{cols=} \FunctionTok{starts\_with}\NormalTok{(}\StringTok{\textquotesingle{}X\textquotesingle{}}\NormalTok{),}\AttributeTok{names\_to =}\StringTok{\textquotesingle{}Date\textquotesingle{}}\NormalTok{,}\AttributeTok{values\_to =} \StringTok{"Case\_Number"}\NormalTok{)}\SpecialCharTok{\%\textgreater{}\%}\FunctionTok{mutate}\NormalTok{(}\AttributeTok{Date=}\FunctionTok{as.Date}\NormalTok{(}\FunctionTok{gsub}\NormalTok{(}\StringTok{\textquotesingle{}X\textquotesingle{}}\NormalTok{,}\StringTok{\textquotesingle{}\textquotesingle{}}\NormalTok{,Date),}\AttributeTok{format=}\StringTok{\textquotesingle{}\%m.\%d.\%y\textquotesingle{}}\NormalTok{))}
\NormalTok{CGL}\OtherTok{\textless{}{-}}\NormalTok{CGL}\SpecialCharTok{\%\textgreater{}\%}\FunctionTok{pivot\_longer}\NormalTok{(}\AttributeTok{cols=} \FunctionTok{starts\_with}\NormalTok{(}\StringTok{\textquotesingle{}X\textquotesingle{}}\NormalTok{),}\AttributeTok{names\_to =}\StringTok{\textquotesingle{}Date\textquotesingle{}}\NormalTok{,}\AttributeTok{values\_to =} \StringTok{"Case\_Number"}\NormalTok{)}\SpecialCharTok{\%\textgreater{}\%}\FunctionTok{mutate}\NormalTok{(}\AttributeTok{Date=}\FunctionTok{as.Date}\NormalTok{(}\FunctionTok{gsub}\NormalTok{(}\StringTok{\textquotesingle{}X\textquotesingle{}}\NormalTok{,}\StringTok{\textquotesingle{}\textquotesingle{}}\NormalTok{,Date),}\AttributeTok{format=}\StringTok{\textquotesingle{}\%m.\%d.\%y\textquotesingle{}}\NormalTok{))}
\NormalTok{DUS}\OtherTok{\textless{}{-}}\NormalTok{DUS}\SpecialCharTok{\%\textgreater{}\%}\FunctionTok{select}\NormalTok{(}\SpecialCharTok{{-}}\NormalTok{UID,}\SpecialCharTok{{-}}\NormalTok{iso2,}\SpecialCharTok{{-}}\NormalTok{iso3,}\SpecialCharTok{{-}}\NormalTok{code3,}\SpecialCharTok{{-}}\NormalTok{FIPS,}\SpecialCharTok{{-}}\NormalTok{Admin2,}\SpecialCharTok{{-}}\NormalTok{Province\_State)}\SpecialCharTok{\%\textgreater{}\%}\FunctionTok{pivot\_longer}\NormalTok{(}\AttributeTok{cols=} \FunctionTok{starts\_with}\NormalTok{(}\StringTok{\textquotesingle{}X\textquotesingle{}}\NormalTok{),}\AttributeTok{names\_to =}\StringTok{\textquotesingle{}Date\textquotesingle{}}\NormalTok{,}\AttributeTok{values\_to =} \StringTok{"Case\_Number"}\NormalTok{)}\SpecialCharTok{\%\textgreater{}\%}\FunctionTok{mutate}\NormalTok{(}\AttributeTok{Date=}\FunctionTok{as.Date}\NormalTok{(}\FunctionTok{gsub}\NormalTok{(}\StringTok{\textquotesingle{}X\textquotesingle{}}\NormalTok{,}\StringTok{\textquotesingle{}\textquotesingle{}}\NormalTok{,Date),}\AttributeTok{format=}\StringTok{\textquotesingle{}\%m.\%d.\%y\textquotesingle{}}\NormalTok{))}
\NormalTok{DGL}\OtherTok{\textless{}{-}}\NormalTok{DGL}\SpecialCharTok{\%\textgreater{}\%}\FunctionTok{pivot\_longer}\NormalTok{(}\AttributeTok{cols=} \FunctionTok{starts\_with}\NormalTok{(}\StringTok{\textquotesingle{}X\textquotesingle{}}\NormalTok{),}\AttributeTok{names\_to =}\StringTok{\textquotesingle{}Date\textquotesingle{}}\NormalTok{,}\AttributeTok{values\_to =} \StringTok{"Case\_Number"}\NormalTok{)}\SpecialCharTok{\%\textgreater{}\%}\FunctionTok{mutate}\NormalTok{(}\AttributeTok{Date=}\FunctionTok{as.Date}\NormalTok{(}\FunctionTok{gsub}\NormalTok{(}\StringTok{\textquotesingle{}X\textquotesingle{}}\NormalTok{,}\StringTok{\textquotesingle{}\textquotesingle{}}\NormalTok{,Date),}\AttributeTok{format=}\StringTok{\textquotesingle{}\%m.\%d.\%y\textquotesingle{}}\NormalTok{))}
\end{Highlighting}
\end{Shaded}

\hypertarget{data-transfer-for-case-fatality-ratio-plot}{%
\subsection{Data Transfer for Case Fatality ratio
plot}\label{data-transfer-for-case-fatality-ratio-plot}}

Summarize the US and Global COIVD-19 case by date. In order to avoid too
few samples and too many fluctuations in the early stage of the
epidemic, we filter the top 10,0000 cases . After this,We citing Case
fatality ratio(CFR) by World Health Organization,to create fatality
ration by date in US and global.
\href{https://www.who.int/news-room/commentaries/detail/estimating-mortality-from-covid-19}{The
World Health Organization states}

\[
  Case\ Fatality\  ratio(CFR,in\%) = \frac{Number\ of\ deaths\ from disease}{Number\ of\ comfirmed\ cases \ of\ individuals} \ \times 100 
\]

We would cumulatively use this CFR function.

\begin{Shaded}
\begin{Highlighting}[]
\NormalTok{USCASE }\OtherTok{\textless{}{-}}\NormalTok{CUS}\SpecialCharTok{\%\textgreater{}\%}\FunctionTok{group\_by}\NormalTok{(Date)}\SpecialCharTok{\%\textgreater{}\%}\FunctionTok{summarise}\NormalTok{(}\AttributeTok{US\_case=}\FunctionTok{sum}\NormalTok{(Case\_Number))}
\NormalTok{GLCASE }\OtherTok{\textless{}{-}}\NormalTok{CGL}\SpecialCharTok{\%\textgreater{}\%}\FunctionTok{group\_by}\NormalTok{(Date)}\SpecialCharTok{\%\textgreater{}\%}\FunctionTok{summarise}\NormalTok{(}\AttributeTok{GL\_case=}\FunctionTok{sum}\NormalTok{(Case\_Number))}
\NormalTok{USDIED}\OtherTok{\textless{}{-}}\NormalTok{DUS}\SpecialCharTok{\%\textgreater{}\%}\FunctionTok{group\_by}\NormalTok{(Date)}\SpecialCharTok{\%\textgreater{}\%}\FunctionTok{summarise}\NormalTok{(}\AttributeTok{USD\_case=}\FunctionTok{sum}\NormalTok{(Case\_Number))}
\NormalTok{GLDIED}\OtherTok{\textless{}{-}}\NormalTok{DGL}\SpecialCharTok{\%\textgreater{}\%}\FunctionTok{group\_by}\NormalTok{(Date)}\SpecialCharTok{\%\textgreater{}\%}\FunctionTok{summarise}\NormalTok{(}\AttributeTok{GLD\_case=}\FunctionTok{sum}\NormalTok{(Case\_Number))}
\NormalTok{USCASE}\OtherTok{\textless{}{-}}\FunctionTok{merge}\NormalTok{(USDIED,USCASE,}\AttributeTok{by=}\StringTok{"Date"}\NormalTok{)}\SpecialCharTok{\%\textgreater{}\%} \FunctionTok{filter}\NormalTok{(US\_case}\SpecialCharTok{\textgreater{}}\DecValTok{100000}\NormalTok{)}
\NormalTok{GLCASE}\OtherTok{\textless{}{-}}\FunctionTok{merge}\NormalTok{(GLDIED,GLCASE,}\AttributeTok{by=}\StringTok{"Date"}\NormalTok{)}\SpecialCharTok{\%\textgreater{}\%}\FunctionTok{filter}\NormalTok{(GL\_case}\SpecialCharTok{\textgreater{}}\DecValTok{100000}\NormalTok{)}
\NormalTok{USCFR}\OtherTok{\textless{}{-}}\NormalTok{USCASE}\SpecialCharTok{\%\textgreater{}\%}\FunctionTok{mutate}\NormalTok{(}\AttributeTok{US\_CFR=}\NormalTok{USD\_case}\SpecialCharTok{*}\DecValTok{100}\SpecialCharTok{/}\NormalTok{US\_case)}
\NormalTok{GLCFR}\OtherTok{\textless{}{-}}\NormalTok{GLCASE}\SpecialCharTok{\%\textgreater{}\%}\FunctionTok{mutate}\NormalTok{(}\AttributeTok{GL\_CFR=}\NormalTok{GLD\_case}\SpecialCharTok{*}\DecValTok{100}\SpecialCharTok{/}\NormalTok{GL\_case)}
\end{Highlighting}
\end{Shaded}

\hypertarget{plotting-us-and-global-case-by-date}{%
\subsection{Plotting US and Global case by
Date}\label{plotting-us-and-global-case-by-date}}

\begin{Shaded}
\begin{Highlighting}[]
\NormalTok{CASE\_plot}\OtherTok{\textless{}{-}}\FunctionTok{ggplot}\NormalTok{()}\SpecialCharTok{+}\FunctionTok{geom\_line}\NormalTok{(}\AttributeTok{data=}\NormalTok{USCASE,}\FunctionTok{aes}\NormalTok{(Date,US\_case,}\AttributeTok{color=}\StringTok{"US case"}\NormalTok{,))}\SpecialCharTok{+}
  \FunctionTok{geom\_line}\NormalTok{(}\AttributeTok{data=}\NormalTok{GLCASE,}\FunctionTok{aes}\NormalTok{(Date,GL\_case,}\AttributeTok{color=}\StringTok{"Global case"}\NormalTok{))}\SpecialCharTok{+}
  \FunctionTok{ylab}\NormalTok{(}\StringTok{"Case"}\NormalTok{)}\SpecialCharTok{+}\FunctionTok{ggtitle}\NormalTok{(}\StringTok{"COVID{-}19 US and Global case"}\NormalTok{)}
\NormalTok{CRF\_plot}\OtherTok{\textless{}{-}}\FunctionTok{ggplot}\NormalTok{()}\SpecialCharTok{+}\FunctionTok{geom\_line}\NormalTok{(}\AttributeTok{data=}\NormalTok{USCFR,}\FunctionTok{aes}\NormalTok{(Date,US\_CFR,}\AttributeTok{color=}\StringTok{"US CFR"}\NormalTok{))}\SpecialCharTok{+}
  \FunctionTok{geom\_line}\NormalTok{(}\AttributeTok{data=}\NormalTok{GLCFR,}\FunctionTok{aes}\NormalTok{(Date,GL\_CFR,}\AttributeTok{color=}\StringTok{"Global CFR"}\NormalTok{))}\SpecialCharTok{+}\FunctionTok{ylab}\NormalTok{(}\StringTok{"CFR"}\NormalTok{)}\SpecialCharTok{+}\FunctionTok{ggtitle}\NormalTok{(}\StringTok{"COVID{-}19 Case fatality ratio(CFR)"}\NormalTok{)}\SpecialCharTok{+}\FunctionTok{scale\_y\_continuous}\NormalTok{(}\AttributeTok{limits =} \FunctionTok{c}\NormalTok{(}\DecValTok{0}\NormalTok{, }\DecValTok{8}\NormalTok{))}
\FunctionTok{plot\_grid}\NormalTok{(CASE\_plot,CRF\_plot,}\AttributeTok{nrow=}\DecValTok{2}\NormalTok{)}
\end{Highlighting}
\end{Shaded}

\includegraphics{COVIDREPORT_files/figure-latex/unnamed-chunk-5-1.pdf}

\hypertarget{tiding-relationship-about-covid-19-cfr-with-latitude-in-usa}{%
\subsection{Tiding Relationship about COVID-19 CFR with latitude in
USA}\label{tiding-relationship-about-covid-19-cfr-with-latitude-in-usa}}

\begin{Shaded}
\begin{Highlighting}[]
\NormalTok{LCase }\OtherTok{\textless{}{-}}\NormalTok{CUS}\SpecialCharTok{\%\textgreater{}\%}\FunctionTok{group\_by}\NormalTok{(Lat)}\SpecialCharTok{\%\textgreater{}\%}\FunctionTok{summarise}\NormalTok{(}\AttributeTok{C\_Lat=}\FunctionTok{sum}\NormalTok{(Case\_Number))}
\NormalTok{LDied}\OtherTok{\textless{}{-}}\NormalTok{DUS}\SpecialCharTok{\%\textgreater{}\%}\FunctionTok{group\_by}\NormalTok{(Lat)}\SpecialCharTok{\%\textgreater{}\%}\FunctionTok{summarise}\NormalTok{(}\AttributeTok{D\_Lat=}\FunctionTok{sum}\NormalTok{(Case\_Number))}
\NormalTok{LCFR}\OtherTok{\textless{}{-}}\FunctionTok{merge}\NormalTok{(LDied,LCase,}\AttributeTok{by=}\StringTok{"Lat"}\NormalTok{)}\SpecialCharTok{\%\textgreater{}\%}\FunctionTok{mutate}\NormalTok{(}\AttributeTok{L\_CFR=}\NormalTok{D\_Lat}\SpecialCharTok{*}\DecValTok{100}\SpecialCharTok{/}\NormalTok{C\_Lat)}
\end{Highlighting}
\end{Shaded}

\hypertarget{ploting-the-relationship-between-cfr-with-latitude}{%
\subsection{Ploting the relationship between CFR with
latitude:}\label{ploting-the-relationship-between-cfr-with-latitude}}

\begin{Shaded}
\begin{Highlighting}[]
\FunctionTok{ggplot}\NormalTok{(LCFR,}\FunctionTok{aes}\NormalTok{(Lat,L\_CFR,}\AttributeTok{color=}\StringTok{"US CFR by Latitdue"}\NormalTok{))}\SpecialCharTok{+}\FunctionTok{geom\_line}\NormalTok{()}
\end{Highlighting}
\end{Shaded}

\includegraphics{COVIDREPORT_files/figure-latex/unnamed-chunk-7-1.pdf}

\hypertarget{summarize-visualization}{%
\subsection{Summarize visualization}\label{summarize-visualization}}

After these two visualization are done,we can see first visualization
show the US CFR is slightly less than the world then gradually
converge.There may be many factors make the CFR under world average,such
as medical level, population structure, epidemic prevention policy. And
we can see the CFR are significant and sustained decline after
2022-July. This may be related to virus mutation or better response plan
for counter COVID-19. According to the second visualization,these data
show Latitude and mortality are roughly linearly independent,but be
caution this CFR may lower average when latitude over 50, it is probably
just a coincidence or partly related temperature, etc.\\

\hypertarget{modeling-us-cfr}{%
\subsection{Modeling US CFR}\label{modeling-us-cfr}}

Assuming that the accumulated by date that known cases can be fitted
with a polynomial function.we can assume: \[
CFR\approx\frac{Case}{b*Case^{1.5}+a*Case+c} \ \times 100 
\]

\begin{Shaded}
\begin{Highlighting}[]
\NormalTok{DATA}\OtherTok{=}\FunctionTok{mutate}\NormalTok{(USCFR,}\AttributeTok{input=}\NormalTok{US\_case}\SpecialCharTok{/}\DecValTok{1000000}\NormalTok{)}
\NormalTok{MODEL }\OtherTok{\textless{}{-}}\FunctionTok{nls}\NormalTok{(US\_CFR }\SpecialCharTok{\textasciitilde{}}\DecValTok{100}\SpecialCharTok{*}\NormalTok{input}\SpecialCharTok{/}\NormalTok{(b}\SpecialCharTok{*}\NormalTok{input}\SpecialCharTok{\^{}}\FloatTok{1.5}\SpecialCharTok{+}\NormalTok{a}\SpecialCharTok{*}\NormalTok{input}\SpecialCharTok{+}\NormalTok{c),}\AttributeTok{start=}\FunctionTok{list}\NormalTok{(}\AttributeTok{a=}\DecValTok{1}\NormalTok{,}\AttributeTok{b=}\DecValTok{1}\NormalTok{,}\AttributeTok{c=}\DecValTok{1}\NormalTok{),}\AttributeTok{data=}\NormalTok{DATA,}\AttributeTok{algorithm =}\StringTok{"port"}\NormalTok{ )}
\FunctionTok{ggplot}\NormalTok{()}\SpecialCharTok{+}\FunctionTok{geom\_line}\NormalTok{(}\AttributeTok{data=}\NormalTok{USCFR,}\FunctionTok{aes}\NormalTok{(Date,US\_CFR,}\AttributeTok{color=}\StringTok{"US CFR"}\NormalTok{))}\SpecialCharTok{+}\FunctionTok{geom\_line}\NormalTok{(}\AttributeTok{data=}\NormalTok{USCASE,}\FunctionTok{aes}\NormalTok{(Date,}\FunctionTok{fitted}\NormalTok{(MODEL),}\AttributeTok{color=}\StringTok{"Fitted function"}\NormalTok{))}
\end{Highlighting}
\end{Shaded}

\includegraphics{COVIDREPORT_files/figure-latex/unnamed-chunk-8-1.pdf}

\begin{Shaded}
\begin{Highlighting}[]
\FunctionTok{summary}\NormalTok{(MODEL)}
\end{Highlighting}
\end{Shaded}

\begin{verbatim}
## 
## Formula: US_CFR ~ 100 * input/(b * input^1.5 + a * input + c)
## 
## Parameters:
##   Estimate Std. Error t value Pr(>|t|)    
## a   0.7775     0.3069   2.533   0.0115 *  
## b  10.9105     0.1241  87.916   <2e-16 ***
## c   5.4590     0.1900  28.734   <2e-16 ***
## ---
## Signif. codes:  0 '***' 0.001 '**' 0.01 '*' 0.05 '.' 0.1 ' ' 1
## 
## Residual standard error: 0.3035 on 865 degrees of freedom
## 
## Algorithm "port", convergence message: relative convergence (4)
\end{verbatim}

This model are good fitted the CFR.We set the maximum degree of this
polynomial to 1.5, which works well. This polynomial hyperparameter may
be related to the rate of infection and the ability of the virus to
mutate.

\hypertarget{bias}{%
\subsection{Bias}\label{bias}}

Firstly,This known COVID-19 cases are not total cases on the world,the
known COVID-19 cases are according to local nucleic acid and testing
policies.In the early days of the epidemic, there were not enough
nucleic acid tests in the world. And now, the rate of voluntary nucleic
acid testing may be lower.when died cases better detection,these cases
would increase the Case fatality ratio,But it will be overestimated than
reality.

Secondly,Testing policies vary around the world,If we get in conclusion
about the US CFR is lower the world,then We can not directly say that
the true death rate is like this.Because the US testing policies may
stricter than other countries,Make the denominator of CFR larger.

In conclusion,This data shows the statistics of COVID-19 in detail, but
we need to summarize more other data such as population, demographics,
temperature, policy impact. to prove the connection between these data
and reality.

\hypertarget{reference}{%
\subsection{Reference:}\label{reference}}

COVID-19 Johns Hopkins github Data source:

\hypertarget{httpsgithub.comcssegisanddatacovid-19treemastercsse_covid_19_datacsse_covid_19_time_series}{%
\subsection{\texorpdfstring{\url{https://github.com/CSSEGISandData/COVID-19/tree/master/csse_covid_19_data/csse_covid_19_time_series}}{https://github.com/CSSEGISandData/COVID-19/tree/master/csse\_covid\_19\_data/csse\_covid\_19\_time\_series}}\label{httpsgithub.comcssegisanddatacovid-19treemastercsse_covid_19_datacsse_covid_19_time_series}}

WHO status Case fatality ratio(CFR):

\url{https://www.who.int/news-room/commentaries/detail/estimating-mortality-from-covid-19}

\end{document}
